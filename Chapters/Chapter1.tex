\chapter{Descrizione del lavoro svolto in campo}
Le esercitazioni strumentali si sono svolte in tre diverse giornate presso il  parco della facoltà di Mesiano, in ciascuna delle quali si è affrontato un diverso tipo di rilievo.
In ogni giornata si sono dapprima viste le modalità di utilizzo degli strumenti e il loro corretto stazionamento.
Al quale è poi seguito un piccolo lavoro di rilievo di alcuni punti concordati e con le opportune convenzioni.

Nelle tre diverse giornate si è affrontato, rispettivamente, l'utilizzo del livello, del GNSS e della stazione totale.

\section{Livello}
L'esercitazione mediante livello si è svolta il giorno 12 marzo 2019 nella zona adiacente al parcheggio multi-piano di Mesiano.
Dopo una rapida spiegazione dell'utilizzo degli strumenti (livello, compensatore e stadie) e della loro messa in stazione, si è svolta l'attività di rilievo. 
L'obiettivo del rilievo era quello di effettuare delle misure di dislivello tra quattro diversi punti e di chiudere l'ultima livellazione con la prima. 
Come modalità operativa si sono utilizzate sia livellazioni dal mezzo - ovvero quelle in cui il livello sta a metà tra due stadie - sia livellazioni reciproche, con il livello prima vicino ad una prima stadia e poi all'altra.  
\section{GNSS}
19 marzo
\section{Stazione totale}
26 

Nei capitoli successivi si riportano i libretti delle misure e le rielaborazioni svolte.