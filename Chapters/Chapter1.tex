\chapter{Descrizione del lavoro svolto in campo}
Le esercitazioni strumentali si sono svolte in tre diverse giornate presso il  parco della facoltà di Mesiano, in ciascuna delle quali si è affrontato un diverso tipo di rilievo.
In ogni giornata si sono dapprima viste le modalità di utilizzo degli strumenti e il loro corretto stazionamento.
Al quale è poi seguito un piccolo lavoro di rilievo di alcuni punti concordati e con le opportune convenzioni.

Nelle tre diverse giornate si è affrontato, rispettivamente, l'utilizzo del livello, del GNSS e della stazione totale. 
Si parlerà ora con maggior dettaglio delle tre tipologie, e nei capitoli successivi si potranno vedere le rielaborazioni svolte con i dati raccolti.
\section{Livello}
L'esercitazione mediante livello si è svolta il giorno 12 marzo 2019 nella zona adiacente al parcheggio multi-piano di Mesiano.
Dopo una rapida spiegazione del funzionamento degli strumenti (livello, micrometro e stadie) e della loro corretta messa in stazione, si è svolta l'attività di rilievo. 

L'obiettivo del lavoro era quello di effettuare delle misure di dislivello tra quattro diversi punti e di chiudere l'ultima livellazione con la prima.  
Come modalità operative si sono utilizzate sia livellazioni dal mezzo -- ovvero quelle in cui il livello sta a metà tra due stadie -- sia livellazioni reciproche, con il livello prima vicino ad una stadia e poi all'altra.
La differenziazione di queste la si nota nei libretti di campagna riportati nel capitolo %inserire riferimento
[pagina o capitolo, boh] e nei calcoli successivi.  
\section{GNSS}
Il giorno 19 marzo 2019 si è svolta l'esercitazione tramite il sistema GNSS 
\section{Stazione totale}
26 
%da tolgiere questo:
Nei capitoli successivi si riportano i libretti delle misure e le rielaborazioni svolte.