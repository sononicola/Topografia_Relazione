\chapter{Descrizione del lavoro svolto}
\label{cap:cap1}
\setcounter{page}{1}
Le esercitazioni strumentali si sono svolte in tre diverse giornate presso il  parco della facoltà di Mesiano, in ciascuna delle quali si è affrontato un diverso tipo di rilievo.
In ogni giornata si sono dapprima viste le modalità di utilizzo degli strumenti e il loro corretto stazionamento.
Al quale è poi seguito un piccolo lavoro di rilievo di alcuni punti concordati e con le opportune convenzioni.

Nelle tre diverse giornate si è affrontato, rispettivamente, l'utilizzo del livello, del GNSS e della stazione totale. 
Si parlerà ora con maggior dettaglio delle tre tipologie, e nei capitoli successivi si potranno vedere le rielaborazioni svolte con i dati raccolti.
\section{Livello}
L'esercitazione mediante livello si è svolta il giorno 12 marzo 2019 nella zona adiacente al parcheggio multi-piano di Mesiano.
Dopo una rapida spiegazione del funzionamento degli strumenti e della loro corretta messa in stazione, si è svolta l'attività di rilievo. 

L'obiettivo del lavoro era quello di effettuare delle misure di dislivello tra quattro diversi punti e di chiudere l'ultima livellazione con la prima. 
Si è perciò iniziato con la messa in stazione dello strumento, consistente nel far sì che l'asse primario fosse verticale e passante per il punto a terra grazie all'ausilio di una livella sferica e del compensatore integrato nel livello.  

Nelle diverse letture alla stadia si è utilizzato anche il micrometro per poter leggere fino alla quarta cifra decimale.
Come modalità operative nei diversi punti si sono utilizzate sia livellazioni dal mezzo -- ovvero quelle in cui il livello sta a metà tra due stadie -- sia livellazioni reciproche, con il livello prima vicino ad una stadia e poi all'altra.
La diversa lettura di queste la si nota nei libretti di campagna riportati nel  capitolo \ref{cap:cap2} e nei calcoli collegati. 

Per ogni punto si sono eseguite più coppie di letture (avanti e indietro), ognuna delle quali fatta da un operatore diverso. Così come la tenuta \emph{in verticale} delle stadie è avvenuta da persone diverse a turno. 
Sì è così venuta a creare una serie di misure ripetute, le quali verranno poi analizzate nei capitoli successivi.
\section{GNSS}
Il giorno 19 marzo 2019 si è svolta l'esercitazione tramite il sistema GNSS. \e avvenuta nel parco di Mesiano, prima nella zona adiacente al laboratorio di Meccanica dei Fluidi, per poi estendersi in altri punti segnati a terra tramite chiodo attorno all'edificio principale.

\e stata fatta prima una spiegazione con una carrellata nei menù dello strumento di tutte le possibili modalità di rilievo fattibili, per poi concentrarsi nella modalità \emph{Real Time Kinematic} (RTK) con la quale sono stati raccolti i dati.
La modalità RTK è stata eseguita appoggiandosi al servizio offerto dalla Provincia Autonoma di Trento \emph{TPOS} grazie il quale è possibile utilizzare come ricevitore una stazione fissa con noti a priori la posizione e i disturbi atmosferici, così da elaborare in tempo reale la posizione del secondo ricevitore.

Il rilievo è consistito nella misura di posizione dei punti prefissati e nell'accertarsi che ci fossero abbastanza satelliti in vista da far risultare un'accuratezza dell'ordine del centimetro. 
\section{Stazione totale}
Il giorno 26 marzo 2019 è avvenuta l'ultima esercitazione, questa volta con l'ausilio di una stazione totale e del prisma per la misura delle distanze. 
Sono stati usati inoltre appositi supporti per il mantenimento in verticale del prisma.
La zona interessata è quella tra il laboratorio e il parcheggio multi-piano di Mesiano. 

L'obiettivo era quello di eseguire letture zenitali e azimutali di tre punti concordati a formare un triangolo e di misurarne la distanza tra essi in andata e ritorno tramite il distanziometro. Un disegno schematico del rilievo è riportato in figura \ref{fig:Triangolo}

La messa in stazione è avvenuta in coincidenza con i tre punti e da ognuna sono stati collimati i due punti rimanenti. 
Anche in questo caso le misure sono state effettuate da operatori diversi. 
Si avranno così misure ridondanti che verranno analizzate nel capitolo \ref{cap:cap4} tramite controlli di compatibilità e compensazione con programma Calge.
\begin{figure}[h]
\centering
\begin{tikzpicture}[>=latex]
	\draw (0,0) -- (5,0) node[right]{$1000$};
	\draw (5,0) -- (4,5) node[above] {$2000$};
	\draw (4,5) -- (0,0)node[left] {$3000$};
\end{tikzpicture}
\caption{Disegno schematico del rilievo con indicazione dei nomi dei punti che si utilizzeranno nel capito \ref{cap:cap4}}
\label{fig:Triangolo}
\end{figure}