\chapter{Livellazione}\label{cap:cap2}
Si riporta innanzitutto il libretto di campagna nella tabella  \ref{tab:libretto1}.
\section{Primo libretto}
%vedere se è davvero il gruppo 3 e BOH
\begin{table}[htb]
\caption{Libretto di campagna del gruppo 3}
\label{tab:libretto1}
\csvreader[centered tabular=|c c|c c|c c|c c|,
	table head=\hline $L_{dx}$ & $L_{sx}$ & $L_{dx}$ %
	& $L_{sx}$ & $L_{dx}$ & $L_{sx}$ & $L_{dx}$ & $L_{sx}$ \\\hline,
	table foot = \hline]%
	{documents/livLibretto1.csv}{}%
	{\csvcoli & \csvcolii & \csvcoliii & %
	\csvcoliv & \csvcolv & \csvcolvi & %
	\csvcolvii & \csvcolviii}
\end{table}
\csvloop{
	file=documents/livLibretto1rielaborato.csv,
	no head,
	before reading=\begin{table}[htb]
	\centering
	\sisetup{table-number-alignment=center}
	\caption{Elaborazione del libretto di campagna assegnando peso 1 a tutte le misure}
	\label{tab:libretto1rielbaborato},
	tabular={|c@{}c|S[table-format=3.3]S[table-format=3.3]|S[table-format=3.3]S[table-format=3.3]|S[table-format=3.3]S[table-format=3.3]|S[table-format=3.3]S[table-format=3.3]|},
	table head=\hline,	
	command={&\csvcolix & \csvcoli & \csvcolii & \csvcoliii & %
	\csvcoliv & \csvcolv & \csvcolvi & %
	\csvcolvii & \csvcolviii},
	table foot=\hline,
	after reading=\end{table}
}
\csvloop{
	file=documents/livLibretto1rielaboratoaggiustato.csv,
	no head,
	before reading=\begin{table}[htb]
	\centering
	\sisetup{table-number-alignment=center}
	\caption{Rielaborazione dopo aver modificato i pesi delle misure}
	\label{tab:libretto1rielbaboratoaggiustato},
	tabular={|c@{}c|S[table-format=3.3]S[table-format=3.3]|S[table-format=3.3]S[table-format=3.3]|S[table-format=3.3]S[table-format=3.3]|S[table-format=3.3]S[table-format=3.3]|},
	table head=\hline,	
	command={&\csvcolix & \csvcoli & \csvcolii & \csvcoliii & %
	\csvcoliv & \csvcolv & \csvcolvi & %
	\csvcolvii & \csvcolviii},
	table foot=\hline,
	after reading=\end{table}
}
\section{Secondo libretto}
\begin{table}[htb]
\caption{Libretto di campagna del gruppo BOH}
\label{tab:libretto2}
\csvreader[centered tabular=|c c|c c|c c|c c|,
	table head=\hline $L_{dx}$ & $L_{sx}$ & $L_{dx}$ %
	& $L_{sx}$ & $L_{dx}$ & $L_{sx}$ & $L_{dx}$ & $L_{sx}$ \\\hline,
	table foot = \hline]%
	{documents/livLibretto2.csv}{}%
	{\csvcoli & \csvcolii & \csvcoliii & %
	\csvcoliv & \csvcolv & \csvcolvi & %
	\csvcolvii & \csvcolviii}
\end{table}
\csvloop{
	file=documents/livLibretto2rielaborato.csv,
	no head,
	before reading=\begin{table}[htb]
	\centering
	\sisetup{table-number-alignment=center}
	\caption{Elaborazione del libretto di campagna assegnando peso 1 a tutte le misure}
	\label{tab:libretto2rielbaborato},
	tabular={|c@{}c|S[table-format=3.3]S[table-format=3.3]|S[table-format=3.3]S[table-format=3.3]|S[table-format=3.3]S[table-format=3.3]|S[table-format=3.3]S[table-format=3.3]|},
	table head=\hline,	
	command={&\csvcolix & \csvcoli & \csvcolii & \csvcoliii & %
	\csvcoliv & \csvcolv & \csvcolvi & %
	\csvcolvii & \csvcolviii},
	table foot=\hline,
	after reading=\end{table}
}
\csvloop{
	file=documents/livLibretto2rielaboratoaggiustato.csv,
	no head,
	before reading=\begin{table}[htb]
	\centering
	\sisetup{table-number-alignment=center}
	\caption{Rielaborazione dopo aver modificato i pesi delle misure}
	\label{tab:libretto2rielbaboratoaggiustato},
	tabular={|c@{}c|S[table-format=3.3]S[table-format=3.3]|S[table-format=3.3]S[table-format=3.3]|S[table-format=3.3]S[table-format=3.3]|S[table-format=3.3]S[table-format=3.3]|},
	table head=\hline,	
	command={&\csvcolix & \csvcoli & \csvcolii & \csvcoliii & %
	\csvcoliv & \csvcolv & \csvcolvi & %
	\csvcolvii & \csvcolviii},
	table foot=\hline,
	after reading=\end{table}
}
