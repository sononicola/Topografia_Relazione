%!TEX root = ../Topografia_Relazione_MeoliNicola.tex
\chapter{Livellazione}\label{cap:cap2}
%LIBRETTO: comando per importare i libretti dal csv
\newcommand{\libretto}[3]{%
\begin{table}[htb]\footnotesize
\caption{#1}
\label{#2}
\centering
\csvreader[centered tabular=c@{}SSSSSSSS,
	table head=\toprule &$\mathbf{L_{dx}}$ & $\mathbf{L_{sx}}$ & $\mathbf{L_{dx}}$ %
	& $\mathbf{L_{sx}}$ & $\mathbf{L_{dx}}$ & $\mathbf{L_{sx}}$ & $\mathbf{L_{dx}}$ & $\mathbf{L_{sx}}$ \\\midrule,
	table foot = \bottomrule]%
	{#3}{}%
	{&\csvcoli & \csvcolii & \csvcoliii & %
	\csvcoliv & \csvcolv & \csvcolvi & %
	\csvcolvii & \csvcolviii}
\end{table}}
%ELABORAZIONE: comando per importare le rielaborazione dei libretti
%1: caption, 2: label, 3: percorso file
\newcommand{\elaborazione}[3]{%
\begin{table}[htbp]\footnotesize
\caption{#1}
\label{#2}
\centering
\csvreader[centered tabular=c@{}rSSSSSSSS,
	no head, %nel csv non c'è la riga iniziale come nei libretti
	table head=\toprule &&$\mathbf{1}$ & $\mathbf{2}$ & $\mathbf{3}$ %
	& $\mathbf{4}$ & $\mathbf{5}$ & $\mathbf{6}$ & $\mathbf{7}$ & $\mathbf{8}$ \\\midrule,
	table foot = \bottomrule]%
	{#3}{}%
	{&\textbf{\csvcolix} & \csvcoli & \csvcolii & \csvcoliii & %
	\csvcoliv & \csvcolv & \csvcolvi & %
	\csvcolvii & \csvcolviii}
\end{table}}
%----------------------------------------
Si elaborano ora i dati raccolti durante il rilievo con il livello di due diversi gruppi. 
Date le diverse misure ripetute dovute alla lettura da parte di diversi operatori dello stesso punto, è necessario analizzare la qualità delle misure. 
Per farlo si calcoleranno medie e varianze al fine di conoscere la dispersione delle misure. 
Quelle eccessivamente discordanti verranno considerate errori grossolani e perciò eliminate. 
Infine si calcolerà la differenza delle letture reciproche e  l'errore di chiusura sommando tutti i dislivelli -- entrambi che dovrebbero risultar zero nel caso ottimale.
\section{Elaborazione del rilievo del gruppo 3}
Si riporta innanzitutto il libretto di campagna nella tabella \ref{tab:libretto1}.
Tabella in cui ogni coppia di righe rappresenta le misure indietro e avanti di uno specifico punto e ogni colonna rappresenta la lettura nella scala destra e sinistra della stadia.
Ogni coppia di colonne è frutto della misura di un diverso operatore. 
%-
\libretto{Libretto di campagna del gruppo 3. Il terzo e quarto insieme di letture sono letture reciproche. Il resto sono letture dal mezzo}{tab:libretto1}{documents/livLibretto1.csv}
%-

Come prima cosa si sono calcolati i dislivelli con la differenza di letture $\Delta_{ij}^{k}=\break l_{indietro}-l_{avanti}$. %break perché altrimenti la spezzava dopo il meno
Si sono così venuti a creare dei dati ridondati di dislivelli per ogni coppia di punti.
Per ogni insieme di dislivelli si è calcolata la media con la quale si è poi calcolata la dispersione di ciascuno di essi.
Come prima battuta si è assegnato un peso pari a $1$ a tutti i dislivelli, come si può vedere nella tabella \ref{tab:libretto1rielaborato} riportata a pagina \pageref{tab:libretto1rielaborato}.
\elaborazione{Elaborazione del libretto di campagna del gruppo 3 assegnando peso 1 a tutti i dislivelli}{tab:libretto1rielaborato}{documents/livLibretto1rielaborato.csv} %
Mentre per i dislivelli con uno scarto troppo eccessivo si è poi modificato il peso portandolo a zero così da eliminarli nel conteggio e ricalcolare poi media e scarti fin tanto da ottenere un valore accettabile come quelli riportati nella tabella \ref{tab:libretto1rielaboratoaggiustato} a pagina \pageref{tab:libretto1rielaboratoaggiustato}.
\elaborazione{Rielaborazione del libretto di campagna del gruppo 3 dopo aver modificato i pesi dei dislivelli}{tab:libretto1rielaboratoaggiustato}{documents/livLibretto1rielaboratoaggiustato.csv}

Nel primo insieme di dislivelli è stato ad esempio prima annullato il dislivello $1$ e $5$ e poi a seguito del ricalcolo della media e degli scarti è stato eliminato anche il dislivello $2$.
Nel secondo insieme sono stati tolti i dislivelli $5$ e $6$, nel quarto insieme soltanto il primo. 
Nel terzo e quinto insieme sono invece stati mantenuti tutti i dislivelli.

Si riportano ora le differenze delle medie delle letture reciproche e le chiusure finale date dalla somma delle medie dei dislivelli. Rispettivamente prima e dopo la rielaborazione con peso modificato.
\begin{align*}
\Delta_{reciproche} &= \si{0.086}{\centi\metre} \quad & chiusura = \si{0.081}{\centi\metre} \\
\Delta_{reciproche} &= \si{-0.242}{\centi\metre} \quad & chiusura = \si{0.076}{\centi\metre}
\end{align*}
Una differenza delle letture reciproche uguale, o comunque non troppo lontana, da zero è indice della corretta rettificazione del livello utilizzato. 
Come si nota nei risultati ottenuti, non solo è abbastanza lontana da zero ma oltretutto aumenta dopo la rielaborazione dei pesi. 
Probabilmente causa del fatto dell'inesperienza degli operatori e di un errore diffuso un po' a tutte le misure.
Discorso analogo per quanto riguarda l'errore di chiusura ma che comunque rappresenta pochi millimetri e perciò accettabile.
%
%\clearpage %mette tutti i float fin qua definiti
\section{Elaborazione del rilievo del gruppo 4}
Lo stesso procedimento appena visto è stato applicato al rilievo fatto da un diverso gruppo. 
Tutti i dati sono riportati con le stesse modalità di prima, perciò si indicheranno soltanto i pesi azzerati e gli errori prima e dopo la rielaborazione. 
\libretto{Libretto di campagna del gruppo 4. Il terzo e quarto insieme di letture sono letture reciproche. Il resto sono letture dal mezzo}{tab:libretto2}{documents/livLibretto2.csv}
\elaborazione{Elaborazione del libretto di campagna del gruppo 4 assegnando peso 1 a tutti i dislivelli}{tab:libretto2rielaborato}{documents/livLibretto2rielaborato.csv}

Ai dislivelli con peso iniziale $1$ della tabella \ref{tab:libretto2rielaborato} sono stati eliminati nel primo insieme i dislivelli $1$ e $2$ in prima battuta e dopo il ricalcolo anche il dislivello $3$. 
Nel secondo insieme è stato azzerato il quarto dislivello.
Per quanto riguarda i dislivelli del terzo insieme è stato eliminato prima il terzo e poi il secondo e l'ottavo.
Nel quarto insieme prima il quinto e poi il sesto dislivello.
Nell'ultimo insieme sono stati mantenuti tutti. 
I dati finali sono riportati nella tabella \ref{tab:libretto2rielaboratoaggiustato} a pagina \pageref{tab:libretto2rielaboratoaggiustato}. 
\elaborazione{Rielaborazione del libretto di campagna del gruppo 4 dopo aver modificato i pesi dei dislivelli}{tab:libretto2rielaboratoaggiustato}{documents/livLibretto2rielaboratoaggiustato.csv}

I calcoli di controllo, rispettivamente prima e dopo la rielaborazione, sono: 
\begin{align*}
\Delta_{reciproche} &= \si{-0.225}{\centi\metre} \quad & chiusura = \si{1.676}{\centi\metre} \\
\Delta_{reciproche} &= \si{0.313}{\centi\metre} \quad & chiusura = \si{0.077}{\centi\metre}
\end{align*}
Si nota come in questo caso si sia partiti da un errore di chiusura elevato fino ad arrivare ad un errore pressoché accettabile.




