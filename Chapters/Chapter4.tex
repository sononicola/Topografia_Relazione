\chapter{Compatibilità e compensazione}
Con i dati raccolti durante le esercitazioni con stazione totale svolte dai vari gruppi, si è proceduto ad eseguire dei controlli di compatibilità tra alcune grandezze plano-altimetriche.
In particolare si sono calcolati i valori mediati delle misure ripetute, controllate le chiusure planimetriche degli angoli e delle distanze e le chiusure altimetriche dei dislivelli.
Si è perciò controllato che la somma degli angoli interni formasse un angolo piatto. 
Che i lati opposti del triangoli rispettassero il teorema dei Carnot, confrontando lato calcolato con il lato misurato. 
E infine che la somma dei dislivelli risultasse zero.

Nel caso di misure palesemente con errori grossolani "a vista" le si è eliminate.
Con le restanti, invece, si è eseguita una compensazione con il programma Calge, confrontando le varianze ed eliminando le misure inadeguate, al fine le coordinate dei punti del triangolo. 
\begin{align*}
	\sum \alpha^{int.} &\cong \si{200}{^g}\\
	c^{mis.} &\cong c^{calc.} = \sqrt{a^2 + b^2 - 2ab\cos(\gamma)} \\
	\sum \Delta_i &\cong 0
\end{align*}

Si riporta di seguito la procedura appena descritta per i dati raccolti dal gruppo 1 e dal gruppo 3 delle esercitazioni. Si riporterà poi un esempio di compensazione eseguito conoscendo le distanze.
\section{Compensazione del rilievo del gruppo 1}

\begin{footnotesize}\centering
\begin{lstlisting}
1000, I, 35.770 , 0, -6.076
2000, I , 20, 30, -4.888
3000, I, 0, 0, 0
$$$$, 0,0,0,0
1000,2000,0,32.859,     392.93545,	102.3237,	1.522,	1.516
1000,3000,0,36.28,      321.55055,	110.56335,	1.600,	1.516
2000,1000,0,32.856,     189.2002,	97.44795,	1.600,	1.522
2000,1000,0,32.856,     189.2017,	97.4526,	1.600,	1.522
2000,3000,0,36.896,     257.8463,	108.5599,	1.467,	1.522
2000,3000,0,36.896,	257.8481,	108.554,	1.467,	1.522
3000,1000,0,36.315,	224.8441,	88.96775,	1.600,	1.467
3000,2000,0,36.896,	164.876,	91.4474,	1.522,	1.467
xxxx,xxxx,0,0,0,0,0,0
GRUPPO UNO
\end{lstlisting}
\end{footnotesize}
%Coordinate gruppo 1
\begin{align*}
	\mathbf{1000}&\quad \si{0.000 \pm 0.001}{} \quad \si{0.000 \pm 0.001}{} \quad \si{0.000 \pm 0.001}{}\\
	\mathbf{2000}&\quad \si{0.000 \pm 0.001}{} \quad \si{0.000 \pm 0.001}{} \quad \si{0.000 \pm 0.001}{}\\
	\mathbf{3000}&\quad \si{0.000 \pm 0.001}{} \quad \si{0.000 \pm 0.001}{} \quad \si{0.000 \pm 0.001}{}
\end{align*}


\section{Compensazione del rilievo del gruppo 3}
\section{Esempio in aula}