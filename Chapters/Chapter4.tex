\chapter{Compatibilità e compensazione}
Con i dati raccolti durante le esercitazioni con stazione totale svolte dai vari gruppi, si è proceduto ad eseguire dei controlli di compatibilità tra alcune grandezze plano-altimetriche.
In particolare si sono calcolati i valori mediati delle misure ripetute, controllate le chiusure planimetriche degli angoli e delle distanze e le chiusure altimetriche dei dislivelli.
Si è perciò controllato che la somma degli angoli interni formasse un angolo piatto. 
Che i lati opposti del triangoli rispettassero il teorema dei Carnot, confrontando lato calcolato con il lato misurato. 
E infine che la somma dei dislivelli risultasse zero.

Nel caso di misure palesemente con errori grossolani "a vista" le si è eliminate.
Con le restanti, invece, si è eseguita una compensazione con il programma Calge, confrontando le varianze ed eliminando le misure inadeguate, al fine le coordinate dei punti del triangolo. 
\begin{align}
	\sum \alpha^{int.} &\cong \si{200}{^g}\\
	c^{mis.} &\cong c^{calc.} = \sqrt{a^2 + b^2 - 2ab\cos(\gamma)} \\
	\sum \Delta_i &\cong 0
\end{align}

Si riporta di seguito la procedura appena descritta per i dati raccolti dal gruppo 1 e dal gruppo 3 delle esercitazioni. Si riporterà poi un esempio di compensazione eseguito conoscendo le distanze.

Obiettivo della compensazione è quello di riuscire ad ottenere risultati adeguati togliendo le misure ripetute che si rilevano troppo discostanti dalle medie, senza però ottenere valori di $\sigma_0$ inferiori all'unità, in quanto poi i risultati finali perderebbero di significato. 

L'ordine di lettura degli input esposti qui di seguito consistono per le prime due comppensazioni in:
\begin{lstlisting}
AAAA , I , X , Y , Z  
......
AAAA , BBBB , J , dist , direz_azim , angolo_zenitale , H_strum_BBBB , H_strum_AAAA
\end{lstlisting}
mentre per l'ultima compensazione in:
\begin{lstlisting}
AAAA , BBBB , K , misura
\end{lstlisting}
In cui con $I$ si intende il parametro per indicare i gradi di libertà di quel punto, con $J$ l'indice per indicare se è accettata quella misura ($0 = $ OK) e con $K$ la tipologia di misura: distanza nel nostro caso.
\section{Compensazione del rilievo del gruppo 1}
EXcel\\
Viene ora affrontata la compensazione vera e propria eseguita con il programma Calge.
Per ottenere le coordinate di partenza da cui far partire il processo di iterazione di Calge, si è assegnata al punto $3000$ l'origine degli assi e si è posto l'orientamento dell'asse $x$ verso il punto $1000$, e assegnato con ascissa pari a $36$. Per le quote si è tenuto conto di un dislivello medio di 4 metri.
All'inizio si sono utilizzate tutte le misure effettuate, che possono essere lette nell'input del programma qui di seguito. 
\begin{lstlisting}
1000, 4, 	36, 		0, 	4
2000, 0 , 	20,	 	30,	4
3000, 1, 	0, 		0, 	0
$$$$, 0,0,0,0
1000,2000,0,32.859,     392.93545,  102.3237,   1.522,  1.516
1000,3000,0,36.28,      321.55055,  110.56335,  1.600,  1.516
2000,1000,0,32.856,     189.2002,   97.44795,   1.600,  1.522
2000,1000,0,32.856,     189.2017,   97.4526,    1.600,  1.522
2000,3000,0,36.896,     257.8463,   108.5599,   1.467,  1.522
2000,3000,0,36.896, 	257.8481,   108.554,    1.467,  1.522
3000,1000,0,36.315, 	224.8441,   88.96775,   1.600,  1.467
3000,2000,0,36.896, 	164.876,    91.4474,    1.522,  1.467
xxxx,xxxx,0,0,0,0,0,0
GRUPPO UNO
\end{lstlisting}
Il quale porta ad un risultato che presenta un residuo troppo elevato nell'angolo zenitale della misura a riga 6.
Si è perciò eliminata tale misura portando ad avere un nuovo residuo nella riga 5 e che ha un $\sigma_0^2$ pari a \si{1.7233}{}.
Si è eliminata anche tale riga, rifacendo il calcolo. 
\e però stato scartato questo nuovo input perché $\sigma_0^2$ è diventata pari a \si{0.1568}{}.

Si è quindi tenuto come risultato finale quello in cui si è eliminata soltanto riga 6 e che porta alle seguenti coordinate dei punti:
%Coordinate gruppo 1
\begin{center}
\begin{tabular}%
		{c%
		S[table-format=2.6]%
		S[table-format=2.6]%
		S[table-format=2.6]}
\toprule
& {$\mathbf{x}$} & {$\mathbf{y}$} & {$\mathbf{z}$}   \\ \midrule
$\mathbf{1000}$ & 35.771665  &  0.000000 &  6.120490 \\
$\mathbf{2000}$ & 21.505886  & 29.570282 &  4.890624 \\
$\mathbf{3000}$ &  0.000000  &  0.000000  & 0.000000 \\
\bottomrule
\end{tabular}
\end{center}
%
\section{Compensazione del rilievo del gruppo 3}
Si riporta ora la compensazione con Calge
\begin{lstlisting}
1000, 4,	36,	0,  	4
2000, 0 ,	20, 	30, 	4
3000, 1, 	0, 	0,   	0
$$$$, 0,0,0,0
1000,2000,0,33.235,	194.6851,	102.32575,	1.408,	1.399
1000,3000,0,36.047,	123.89465,	110.4399,	1.600,	1.399
2000,1000,0,33.243,	26.779,		97.2715,	1.602,	1.408
2000,1000,0,33.244,	26.78865,	97.27255,	1.602,	1.408
2000,3000,0,36.684,	94.83175,	108.4262,	1.441,	1.408
2000,3000,0,36.684,	94.8322,	108.42515,	1.441,	1.408
3000,1000,0,36.097,	330.8344,	88.8924,	1.602,	1.441
3000,2000,0,36.684,	269.7198,	91.57075,	1.408,	1.441
xxxx,xxxx,0,0,0,0,0,0
GRUPPO TRE
\end{lstlisting}
Utilizzando il qui sopra esposto libretto si ottiene un residuo elevato nella seconda misura. 
Si è perciò eliminata la riga 6 portando ad un $\sigma_0^2$ pari a \si{2.1511}{}.
Facendo varie prove si è notato che eliminando altre righe comportava ad avere $\sigma_0^2$ troppo bassi, pertanto si è mantenuto il risultato ottenuto avendo tolto solo la seconda misura (riga 6 dell'input)

In questo caso si sono ottenuti valori di varianze più elevate rispetto al gruppo 1 e soprattutto non errori non attribuibili ad una specifica misura.
Tale considerazione la si può vedere confrontando i valori delle singole coordinate. 

Si ottengono pertanto le seguenti coordinate dei punti:
%Coordinate gruppo 1
\begin{center}
\begin{tabular}%
		{c%
		S[table-format=2.6]%
		S[table-format=2.6]%
		S[table-format=2.6]}
\toprule
& {$\mathbf{x}$} & {$\mathbf{y}$} & {$\mathbf{z}$}   \\ \midrule
$\mathbf{1000}$ & 35.544380 & 0.000000 & 6.106777 \\
$\mathbf{2000}$ & 20.855675 &29.788713 & 4.876273 \\
$\mathbf{3000}$ &  0.000000 & 0.000000 & 0.000000 \\
\bottomrule
\end{tabular}
\end{center}

%

\section{Esempio in aula}
\begin{lstlisting}
1000, 1, 	1.2, 		0,0
2000, 1 , 	0,	 	0,0
3000, 0, 	0, 		0,0
$$$$, 0,	0,	0,0
1000, 3000,	0,	1.0220290
2000,3000,	0,	1.0102192
xxxx,xxxx,	0,	0
TRIANGOLO
\end{lstlisting}
