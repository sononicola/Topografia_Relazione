\chapter{Compensazione con CALGE}
\section{Esempio in aula}
\section{Compensazione del rilievo del gruppo 1}
\begin{footnotesize}\centering
\begin{lstlisting}
1000, I, 35.770 , 0, -6.076
2000, I , 20, 30, -4.888
3000, I, 0, 0, 0
$$$$, 0,0,0,0
1000,2000,0,32.859,     392.93545,	102.3237,	1.522,	1.516
1000,3000,0,36.28,      321.55055,	110.56335,	1.600,	1.516
2000,1000,0,32.856,     189.2002,	97.44795,	1.600,	1.522
2000,1000,0,32.856,     189.2017,	97.4526,	1.600,	1.522
2000,3000,0,36.896,     257.8463,	108.5599,	1.467,	1.522
2000,3000,0,36.896,	257.8481,	108.554,	1.467,	1.522
3000,1000,0,36.315,	224.8441,	88.96775,	1.600,	1.467
3000,2000,0,36.896,	164.876,	91.4474,	1.522,	1.467
xxxx,xxxx,0,0,0,0,0,0
GRUPPO UNO
\end{lstlisting}
\end{footnotesize}
%Coordinate gruppo 1
\begin{align*}
	\mathbf{1000}&\quad \si{0.000 \pm 0.001}{} \quad \si{0.000 \pm 0.001}{} \quad \si{0.000 \pm 0.001}{}\\
	\mathbf{2000}&\quad \si{0.000 \pm 0.001}{} \quad \si{0.000 \pm 0.001}{} \quad \si{0.000 \pm 0.001}{}\\
	\mathbf{3000}&\quad \si{0.000 \pm 0.001}{} \quad \si{0.000 \pm 0.001}{} \quad \si{0.000 \pm 0.001}{}
\end{align*}


\section{Compensazione del rilievo del gruppo 3}